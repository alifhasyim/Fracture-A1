\documentclass[12pt]{article}
\usepackage{graphicx}
\usepackage{caption}
\usepackage{amsmath}
\usepackage{geometry}
\usepackage{float}
\usepackage{booktabs}
\usepackage{longtable}
\usepackage{setspace} % for spacing control
\usepackage{indentfirst}
\usepackage{titlesec}
\usepackage{listings}
\usepackage{xcolor}
\usepackage{pifont}
\usepackage{tcolorbox}
\usepackage[backend=biber,style=numeric]{biblatex} 
\addbibresource{bibliography_CIBB_file.bib}
\setstretch{1.0}       % single line spacing (default)
\setlength{\parskip}{0pt}   % no space between paragraphs
\setlength{\parindent}{0em} % set indentation amount

\makeatletter

\AtBeginDocument{%
  % Counter `lstlisting' is not defined before `\begin{document}'
  \newcounter{llabel}[lstlisting]%
  \renewcommand*{\thellabel}{%
    \ifnum\value{llabel}<0 %
      \@ctrerr
    \else
      \ifnum\value{llabel}>10 %
        \@ctrerr
      \else
        \protect\ding{\the\numexpr\value{llabel}+201\relax}%
      \fi
    \fi
  }%
}
\newlength{\llabelsep}
\setlength{\llabelsep}{5pt}

\newcommand*{\llabel}[1]{%
  \begingroup
    \refstepcounter{llabel}%
    \label{#1}%
    \llap{%
      \thellabel\kern\llabelsep
      \hphantom{\lst@numberstyle\the\lst@lineno}%
      \kern\lst@numbersep
    }%
  \endgroup
}
\makeatother

\geometry{margin=1in}
\setlength{\parskip}{0em}
\setlength{\parindent}{0pt}

\title{Assignment 1 \\ \large Computational Fracture Mechanics (WiSe2526)}
\author{Bagus Alifah Hasyim \\ 108023246468}
\date{}

\begin{document}
\maketitle

\section*{Given Data}
\hspace*{2em}In this section, the parameters value for the material properties, 
that required by the author to calculate it based on the author's Immatrikulation Nummer, will be defined accordingly. To make
the calculation detailed and transparent, each parameter will be derived step by step. 

\begin{equation}
    d_1 = 4, \text{   }\text{   } d_2 = 6, \text{   }\text{   } d_3 = 8
\end{equation}

\vspace{1em}
\begin{tcolorbox}[title=Parameter for 1st question]
  \begin{itemize}
    \item Plate length b
    
    $b = 2.0 + 0.1d_1 = 2.0 + 0.1 \times 4 = 2.4 \text{ mm}$
    \item Plate width h 
    
    $h = b(0.4 + 0.04d_2) = 2.4(0.4 + 0.04 \times 6) = 1.536 \text{ mm}$
    \item Crack length l
    
    $l = b(0.05 + 0.02d_3) = 2.4(0.05 + 0.02 \times 8) = 0.504 \text{ mm}$
    \item Applied displacement u 
    
    $u = 0.001 + \frac{d_3}{1000} [\text{mm}] = 0.001 + \frac{8}{1000} = 0.009 \text{ mm}$  
\end{itemize}
\end{tcolorbox}


\vspace{1em}
\begin{tcolorbox}[title=Parameter for 2nd question]
  \begin{itemize}
    \item Applied displacement u
    $u = 0.1 + \frac{d_3}{100}[\text{mm}] = 0.1 + \frac{8}{100} = 0.18 \text{ mm}$ 
\end{itemize}
\end{tcolorbox}
\vspace{1em}


\section{Introduction} 
\hspace{2em}In this first assignments, the main focus will be on performing crack analysis
using finite element method (FEM) in Abaqus software using two approaches. The first
method is to perform a pre-defined crack analysis without separation between two materials,
whereas the area around the crack tip will be evaluated to obtain the behavior of the stress field. Hence, J-integral
will be calculated based on the stress field using domain integral method to obtain the energy release quantity. The key takeaway from this
is to find the stable value of J-integral based on the contour integral surrounding the crack, which will be used
as a reference value for the crack initiation value.

\hspace{2em}The second task is to analyze crack propagation using the cohesive zone method. Hence, some
cohesive elements will be defined along the pre-defined crack path with the 
cohesive parameters with an initial crack opening on the front of the cohesive line. 
From the task definition, the separation behavior of the cohesive are will be observed, as well as the energy release 
rate. This can be achieved by plotting the traction-separation curve, which will show the
relationship between the traction and separation displacement along the cohesive zone. Another key aspect that will be obtained
is to know the force-time response during the crack propagation, so that it can be correlated with the traction-separation
curve.


\section{Set up of the Model}

\subsection{Material Properties}

\begin{table}[H]
\centering
\begin{minipage}{0.49\textwidth}
\centering
\caption{Flow curve data for Q1 and Q2}
\label{tab:PlasticityQ1Q2}
\begin{tabular}{cc}
    \toprule
    Plastic Stress (MPa) & Strain \\
    \midrule
    300 & 0 \\
    400 & 0.001 \\
    600 & 0.003 \\
    700 & 0.005 \\
    \toprule
    Young Modulus (GPa) & $\nu$ \\
    \midrule
    210 & 0.3 \\
    \bottomrule
\end{tabular}
\end{minipage}
\hfill
\begin{minipage}{0.49\textwidth}
\centering
\caption{Cohesive parameters for Q2}
\label{tab:CohesiveQ2}
\begin{tabular}{lc}
    \toprule
    Material Parameters & Values \\
    \midrule
    $K_n$ (GPa) & 210 \\
    $K_s$ (GPa) & 210 \\
    $K_t$ (GPa) & 210 \\
    $\sigma_n$ (MPa) & 500 \\
    $\sigma_{s,t}$ & 0 \\
    $\delta_{fail}$ (mm) & 0.0038 \\
    \bottomrule
\end{tabular}
\end{minipage}
\end{table}


\vspace{1em}
\hspace{2em}Table~\ref{tab:PlasticityQ1Q2} shows the material properties for the elastic and plastic part, which includes the plastic flow with its strain as well
as the young modulus and Poisson's ratio for the elastic part. Here, the material 
will be used for both question 1 and question 2. Meanwhile, table~\ref{tab:CohesiveQ2} shows the cohesive parameters, which will be used to define
the cohesive zone model in question 2. The parameters include the stiffness of the cohesive area in normal and shear 
directions, the maximum nominal stress, and the failure separation displacement.



\subsection{Geometry and Boundary Conditions}
\begin{figure}[H]
    \centering
    \includegraphics[width=0.9\textwidth]{images/GeoDef.png}
    \caption{Geometry and boundary definition of the plate with horizontal crack on the left-middle side of the plate. The geometry parameters are defined
    in the first section of the report, whereas b = 2.4 mm, h = 1.536 mm, and l = 0.504 mm. This dimension configuration is used for both
    question 1 and 2.}
    \label{fig:GeoDef}
\end{figure}

\hspace{2em}Figure~\ref{fig:GeoDef} shows the geometry and boundary conditions of the plate with horizontal crack. Hence, the displacement will
be applied on the left side of the plate, whereas the left-top edge will have displacement to the y-positive direction and left-bottom edge
will have the opposite direction. The right side of the plate will be fixed in both x and y directions. For the first task,
a crack with length of $l$ is defined on the left-middle side of the plate. This definition
will ensure a separation, meaning that the nodes are separated with this crack line.

\hspace{2em}For the second task, cohesive area is defined along the C length from the crack tip until the
edge of the right side of the plate. To achieve this, two separate geometry are created. This consist of top and bottom plate,
which then both of them is connected using contact definition. Hence, the contact is defined as a cohesive contact, 
where it spans through the C line according to figure~\ref{fig:GeoDef}. The cohesive parameters are defined as in table~\ref{tab:CohesiveQ2}.
Since the cohesive area is defined only in the C line, therefore the non-contact area (hence is $l_{crack}$) is consider
as the crack initiation length.

\subsection{Mesh Set Up}

\hspace{2em}The next step after setting up the geometry and boundary condition is the mesh definition. Since there is
no constraint in terms of element numbers as well as computational power restriction, a fine mesh is used for both questions.

\begin{figure}[H]
    \centering
    \includegraphics[width=0.8\textwidth]{images/Mesh_Task1.png}
    \caption{Mesh configuration for the first task. A standard and linear CPS4 (4-node bilinear plane stress quadrilateral) mesh configuration is implemented 
    with the possibility to change to plane stress and plane strain condition, without reduced integration mode. Additional refinement 
    is added to the crack tip with more structured mesh wit 15 nodes mesh partitioning.}
    \label{fig:Mesh_Task1}
\end{figure}

\begin{figure}[H]
    \centering
    \includegraphics[width=0.8\textwidth]{images/Mesh_Task2.png}
    \caption{Final mesh configuration for both question 1 and question 2. The mesh is created using CPS4 element type, which is a 4-node bilinear plane stress quadrilateral element.
    The mesh is refined around the crack tip area to accurately capture the stress concentration effects.}
    \label{fig:Mesh_Task2}
\end{figure}



\newpage
\section*{Results and Discussion}

%% ------------------------------------------------------------- %%
%% ----------------------------- %% QUESTION 1 ----------------------------- %%
%% ------------------------------------------------------------- %%

\subsection*{Q1: Crack in plate under uniaxial tension}

\vspace{1em}
\begin{tcolorbox}[title=Elastic Analysis]
\textit{$\dots$ Using the elastic material parameters, 
generate a contour plot of the von Mises stress near the crack 
tip. Extract and plot the relevant stress components
along path C (see Figure 1) for both plane strain and plane 
stress conditions. Compare and discuss the differences between 
the two cases.}
\end{tcolorbox}


\begin{figure}[H]
    \centering
    \begin{minipage}{0.49\textwidth}
        \centering
        \includegraphics[width=\textwidth]{images/Q1_Elastic_PStress.png}
        \caption{Von Mises stress contour plot around the crack tip for elastic analysis under plane stress condition.}
        \label{fig:Q1_Elastic_PStress}
    \end{minipage}
    \hfill
    \begin{minipage}{0.49\textwidth}
        \centering
        \includegraphics[width=\textwidth]{images/Q1_Elastic_PStrain.png}
        \caption{Von Mises stress contour plot around the crack tip for elastic analysis under plane strain condition.}
        \label{fig:Q1_Elastic_PStrain}
    \end{minipage}
\end{figure}



\begin{figure}[H]
    \centering
    \includegraphics[width=0.99\textwidth]{images/Q1_Elastic_Comparison.png}
    \caption{Stress components along path C for both plane stress and plane strain conditions in elastic analysis.}
    \label{fig:Q1_Elastic_Stress_Curve}
\end{figure}

\newpage
\vspace{1em}
\begin{tcolorbox}[title=Elastic-plastic analysis]
\textit{$\dots$ Now include plasticity using the material parameters given in Table 1. 
Produce a contour plot of the von Mises stress near the crack tip. Extract and plot the relevant 
stress and strain components along path C for both plane strain and plane stress conditions. Discuss:
\begin{itemize}
\item The differences between the elastic and elastic-plastic responses
\item The differences in the size and shape of the stress field around the crack tip for plane strain vs. plane stress
\end{itemize}}
\end{tcolorbox}

\begin{figure}[H]
  \centering
  \includegraphics[width=0.99\textwidth]{images/Q1_ResponseElPl.png}
  \caption{Von Mises stress comparison between elastic and elastic-plastic analysis under plane stress condition.}
  \label{fig:Q1_ResponseElPl}
\end{figure}



\begin{figure}[H]
    \centering
    \begin{minipage}{0.49\textwidth}
        \centering
        \includegraphics[width=\textwidth]{images/Q1_Plastix_PStress.png}
        \caption{Von Mises stress contour plot around the crack tip for plastic analysis under plane stress condition.}
        \label{fig:Q1_Plastic_PStress}
    \end{minipage}
    \hfill
    \begin{minipage}{0.49\textwidth}
        \centering
        \includegraphics[width=\textwidth]{images/Q1_Plastix_PStrain.png}
        \caption{Von Mises stress contour plot around the crack tip for plastic analysis under plane strain condition.}
        \label{fig:Q1_Plastic_PStrain}
    \end{minipage}
\end{figure}



\begin{figure}[H]
    \centering
    \includegraphics[width=0.99\textwidth]{images/Q1_Comparison_All.png}
    \caption{Stress components along path C for both plane stress and plane strain conditions in elastic-plastic analysis.}
    \label{fig:Q1_C_StressCurve}
\end{figure}

\begin{figure}[H]
    \centering
    \includegraphics[width=0.99\textwidth]{images/Q1_Comparison_Img_All.png}
    \caption{Stress components along path C for both plane stress and plane strain conditions in elastic-plastic analysis.}
    \label{fig:Q1_C_StressCurve}
\end{figure}

\newpage
\vspace{1em}
\begin{tcolorbox}[title=J-integral evaluation]
\textit{$\dots$ For both elastic and elastic-plastic cases, 
evaluate the J-integral over 10 contours around the crack tip only 
under plane stress condition. Plot the J-integral value as a function of 
contour number and use this plot justify which contour range you consider 
reliable for reporting the final J value.
\begin{itemize}
    \item For the purely elastic material model under plane stress condition, compute the J-integral
    \item Repeat the J-integral computation including plasticity. ($\alpha$ = 0.01, n = 10)
    \item Compare the J-integral results from pure elastic material model and elastic-plastic model, discuss the differences based on 
    the theoretical J-integral calculation methods given by the lecture.
\end{itemize}
}
\end{tcolorbox}

\vspace{1em}
\hspace{1em}In this section, J-integral values are evaluated for both elastic and elastic-plastic cases in plain stress
condition. To define the elastic-plastic behavior, Ramberg-Osgood flow curve is generated to describe the Plasticity
region in the flow curve. Ramberg-Osgood relation is defined as following equation

\begin{equation}
    \varepsilon = \frac{\sigma}{E} + \alpha \left( \frac{\sigma}{\sigma_y} \right)^n
    \label{eq:RambergOsgood}
\end{equation}

\hspace{1em}Where $\varepsilon$ is the total strain, $\sigma$ is the stress, 
$E$ is the Young's modulus, $\sigma_y$ is the yield stress, n is the strain hardening exponent,
and $\alpha$ is the coefficient for controlling the yield curve transition. To generate the flow curve, equation
~\ref{eq:RambergOsgood} is rearranged to find the stress value based on the strain value, 
which is then plotted to obtain the flow curve as in figure~\ref{fig:RambergOsgood}. The modified equation, which the values are directly
inserted to Abaqus material parameters definition, can be written as follows:

\begin{equation}
    \sigma = \sigma_y \left( \frac{\varepsilon}{\alpha \varepsilon_y} \right)^{\frac{1}{n}}
    \label{eq:modifiedRambergOsgood}
\end{equation}

\hspace{1em}Important note that the yield stress is defined as 300 MPa based on table~\ref{tab:PlasticityQ1Q2}, hence the yield strain
can be calculated as follows:

\begin{equation}
    \varepsilon_y = \frac{\sigma_y}{E} = \frac{300 \text{ MPa}}{210000 \text{ MPa}} = 0.00142857
    \label{eq:yieldStrain}
\end{equation}

\begin{figure}[H]
\centering
\includegraphics[width=0.99\textwidth]{images/Ramberg-Osgood.png}
\caption{Ramberg-Osgood flow curve based on the Ramberg-Osgood equation with $\alpha$ = 0.01 and n = 10}
\label{fig:RambergOsgood}
\end{figure}

After the elastic-plastic behavior is defined, the J-integral values are evaluated for both
elastic and elastic-plastic cases in plane stress condition. The J-integral values results as well as the von Mises stress comparison are plotted for elastic and elastic-plastic condition
in figure~\ref{fig:Q1_JIntegral}. From the figure,
it can be observed that J-integral values for both cases show a converging trend as the contour number increases.
The main difference is highlighted in the different values of J-integral, as the elastic-plastic case have smaller values compare to
elastic case, which shown in shifting downward of the elastic-plastic curve. This can be correlated with the von Mises stress
distribution around the crack tip, where the elastic case shows higher stress concentration compare to the elastic-plastic case.

\begin{figure}[H]
  \centering
  \includegraphics[width=0.99\textwidth]{images/Q1_JIntegral.png}
  \caption{J-integral values over 10 contours around the crack tip for both elastic and elastic-plastic analysis under plane stress condition. The measurement
  technique used is domain integral method, as the contour integral is evaluated from n=1 until n=10.}
  \label{fig:Q1_JIntegral}
\end{figure}

To validate the J-integral results, evaluating the theoretical J-integral value can be done so that the accuracy of the FEM
simulation can be known. The first step is by knowing the stress intensity factor $K_I$ for the crack tip, '
which can be calculated using the following equation:

\begin{equation}
    K_I = \frac{P}{B \sqrt{W}} f\left( \frac{a}{W} \right)
    \label{eq:StressIntensityFactor}
\end{equation}

Where P is the applied load, B is the thickness (considered as 1 mm for plane stress condition), W is the width of the plate,
and a is the crack length. The function $f\left( \frac{a}{W} \right)$ is a geometry correction factor, which can be calculated using the following equation:

\begin{equation} 
  \begin{split} f\!\left(\frac{a}{W}\right) &
    = \frac{\,2 + \frac{a}{W}\,}{\left(1 - \frac{a}{W}\right)^{3/2}} 
    \Bigl( 0.886 + 4.64\frac{a}{W} - 13.32\left(\frac{a}{W}\right)^2 \\ 
    &\qquad + 14.72\left(\frac{a}{W}\right)^3 - 5.60\left(\frac{a}{W}\right)^4 \Bigr) 
  \end{split} 
  \label{eq:GeometryCorrectionFactor} 
\end{equation}

\hspace{1em}This equation is obtained based on literature Zhu et al. (2010) for compact C(T) specimen under mode I loading condition~\cite{Zhu.2012}, hence f = 6.043. The missing puzzle located
in the loading P, which here based on the literature is a singular load applied in the hole area of the C(T) specimen. To obtain this equivalent load,
the reaction force from the applied displacement is extracted from the FEM simulation by probing the reaction force in the y-direction
on the left edge of the plate. Based on the probing values in Abaqus, the reaction force is obtained as 205.559 N for the elastic case, while 
for elastic-plastic case the observed reaction force is 97.5102 N.

\hspace{1em}After the stress-intensity factor is obtained, the theoretical J-integral value can be calculated. Noted from the literature~\cite{Zhu.2012}, the J integral
is distinguished based on the material behavior. For elastic case, J-integral can be calculated using the following equation:

\begin{equation}
    J_{el} = \frac{K_I^2}{E'}
    \label{eq:JIntegralElastic}
\end{equation}

Where $E'$ is the effective Young's modulus, which for plane stress condition is equal to E. Meanwhile, for elastic-plastic case, J-integral can be calculated using the following equation:
\begin{equation}
    J = J_{el} + J_{pl} = \frac{K_I^2}{E'} + \frac{\eta_{pl}A_{pl}}{Bb} 
    \label{eq:JIntegralElasticPlastic}
\end{equation}

Where $\eta_{pl}$ is a geometry factor, $A_{pl}$ is the plastic area, $B$ is the thickness, and $b$ is the uncracked ligament length ($b = W - a$). Hence, $A_{pl}$ will be calculated manually by observing
the plastic area by looking at the PEEQ. Based on the measurement, the plastic area for the elastic-plastic specimen is 7.333e-02 $mm^2$. While the geometry factor can be calculated using reference of Clarke et.al~\cite{Clarke.1979}. The geometry factor
can be calculated using the following equation:
\begin{equation}
    \eta_{pl} = 2 + 0.522 \frac{b}{W}
    \label{eq:GeometryFactor}
\end{equation}

A table is created to summarize the theoretical J-integral values calculation for both elastic and elastic-plastic case. The summary
is shown in table~\ref{tab:JIntegralCalculation}.

\begin{table}[H]
\centering
\caption{Theoretical J-integral value calculation summary for elastic and elastic-plastic case.}
\label{tab:JIntegralCalculation}
\resizebox{\textwidth}{!}{%
\begin{tabular}{lccccccc}
    \toprule
    Material Model & P (N) & $K_I$ (MPa$\sqrt{m}$) & $J_{el}$ (N/mm) & $A_{pl}$ (mm$^2$) & $\eta_{pl}$ & $J_{pl}$ (N/mm) & Total J (N/mm) \\
    \midrule
    Elastic & 205.559 & 1000.2902 & 4.765 & -- & -- & -- & 4.765 \\
    Elastic-Plastic & 97.5102 & 475.452 & 1.076 & $7.333\times10^{-2}$ & 2.351 & 0.167 & 1.243 \\
    \bottomrule
\end{tabular}}
\end{table}



%% ------------------------------------------------------------- %%
%% ----------------------------- %% QUESTION 2 ----------------------------- %%
%% ------------------------------------------------------------- %%

\newpage
\subsection*{Q2: Crack propagation using a Cohesive Surface}

\vspace{1em}
\begin{tcolorbox}[title=Traction-separation law]
  \textit{$\dots$ Construct the traction-separation curve using the cohesive parameters in Table 2. 
Calculate the fracture energy from this curve and briefly describe the meaning of each segment.}
\end{tcolorbox}

\vspace{1em}
\hspace{1em}The first step to construct the traction-separation curve is to determine the known parameters, which has been defined
in table~\ref{tab:CohesiveQ2}. Hence, the stiffness values in normal and shear direction are
$K_n = K_s = K_t = 210 \text{ GPa} = 210000 \text{ MPa}$. The maximum nominal stress in normal direction is
$\sigma_n = 500 \text{ MPa}$, while the shear direction maximum nominal stress is $\sigma_{s,t} = 0 \text{ MPa}$. The failure separation displacement
is $\delta_{fail} = 0.0038 \text{ mm}$. This information is then used to draw the curve, by plotting the stiffness as the
slope of the curve in the initial linear region, until it reaches the maximum nominal stress. After knowing the maximum
value, a line is drawn from the maximum nominal stress to the failure separation displacement point (hence the stress value is 0),
which represents the softening behavior of the cohesive zone. The final traction-separation curve is shown in figure~\ref{fig:TractionSeparationCurve}.

\begin{figure}[H]
  \centering
  \includegraphics[width=0.99\textwidth]{images/Q2_TractionSeparationCurve.png}
  \caption{Traction-separation curve based on the cohesive parameters defined in table~\ref{tab:CohesiveQ2}. The curve shows the initial linear elastic region,
  followed by the softening region until failure separation displacement is reached.}
  \label{fig:TractionSeparationCurve}
\end{figure}

\newpage
\vspace{1em}
\begin{tcolorbox}[title=Crack propagation plots]
  \textit{$\dots$ Include a figure of the selected mesh and specify the mesh size used in the simulation. 
Plot the relevant stress components at several load increments to clearly show how the crack initiates 
and propagates along the cohesive surface.}
\end{tcolorbox}

\begin{figure}[H]
  \centering
  \includegraphics[width=0.99\textwidth]{images/Q2_CrackPropagarionPlot.png}
  \caption{Crack propagation plot showing the von Mises stress distribution at several load increments. The crack initiates and propagates along the cohesive surface as the load increases.}
  \label{fig:Q2_CrackPropagationPlot}
\end{figure}

\newpage
\vspace{1em}
\begin{tcolorbox}[title=Force-time response]
  \textit{$\dots$ Compute the total reaction force in the y-direction on the left edge of the upper part. 
Plot force versus time and summarize how the slope changes as the crack initiates and propagates.}
\end{tcolorbox}

\begin{figure}[H]
  \centering
  \includegraphics[width=0.99\textwidth]{images/Q2_ForceTimeResponse.png}
  \caption{Force-time response showing the total reaction force in the y-direction on the left edge of the upper part. The slope changes indicate crack initiation and propagation.}
  \label{fig:Q2_ForceTimeResponse}
\end{figure}

\newpage
\vspace{1em}
\begin{tcolorbox}[title=Cohesive Surface Analysis]
  \textit{$\dots$ Extract traction values at three selected points along the cohesive surface. 
Compare these results with the analytical traction-separation curve and provide a concise conclusion.}
\end{tcolorbox}




\newpage
\section*{Conclusion}
To sum up everything that have been analyzed and gathered in this study, 
here are some several key takeaways from this small case study:

\printbibliography

\end{document}

